%% -*- fill-column: 120; -*-
\section{Prior Work}
\label{sec_priorart}

% -----------------------------------------------------------------
% -----------------------------------------------------------------
\subsection{Distributed Consensus}
\subsubsection{Byzantine Fault Tolerance}

Describe Byzantine Generals problem.

Describe PBFT family of algorithms

The Good:
\begin{itemize}
\item Performant
\item Finality
\end{itemize}

The Challenges:
\begin{itemize}
\item Lots of communication
\item Sensitive to network conditions
\item Identity of all participants must be known
\end{itemize}

Limited decentralization.

% -----------------------------------------------------------------
% -----------------------------------------------------------------
\subsubsection{Nakamoto Consensus}

Describe Nakamoto Consensus as implemented in Bitcoin.

Replace finality with eventual consistency (probabalistic)

Proof of work as leader election and sybil prevention.

Implicit voting.

The Good:
\begin{itemize}
\item Decentralization of authority and implementation
\end{itemize}

The Challenges:
\begin{itemize}
\item Proof of work is extremely expensive
\end{itemize}

Properties of leader election in Nakamoto Consensus
\begin{itemize}
\item Uses distribution in time to optimize communication. A good leader
  election algorithm distributes election over time to increase the
  probability that only one broadcast is necessary.
\item Resistant to censorship
\item Adapts to population changes
\end{itemize}

% -----------------------------------------------------------------
% -----------------------------------------------------------------
\subsection{Trusted Execution}
We love SGX.

\begin{itemize}
\item Confidentiality--Host cannot see what is executed in the enclave
\item Integrity--The software has not been tampered with
\item Attestation--Proof that we ran what we expected
\end{itemize}

The basic paradigm
\begin{enumerate}
\item Perform a computation that operates on some input and generates some output
\item Prove to others that:
\begin{itemize}
\item The computation was performed by the agreed upon code
\item For the given input the computation generated the given output
\end{itemize}
\end{enumerate}
