%% -*- fill-column: 120; -*-
\section{Introduction}
\label{sec_introduction}

Over the last several years, cryptographic currencies like Bitcoin~\cite{} started a disruption that has grown
well-beyond currency. Inspired by the decentralization of fiduciary authority represented by Bitcoin, businesses and
governments across the world are investigating disintermediated and decentralized services. Usages are diverse:
federated identity, clearing and settlement of financial transactions, multi-party supply chain management, provenance
of food, and privacy-preserving access to medical records. At the core of this revolution is a data structure called a
{\it blockchain}. Simply put, a blockchain is a data structure that uses cryptographic hashes to create an immutable
ordering of events. Communities use blockchains to represent a shared view of ``truth''. For example, when events like
``transfer $50 from john to jane'' or ``grant tom access to file A'' are committed to a blockchain, the entire community
agrees that the event occurred and can act accordingly.

There are many ways for a community to build and maintain a blockchain. In some cases, a trusted organization (like a
representative consortium or a contracted infrastructure provider) can build a blockchain using fairly traditional
distributed database technologies. However, for those cases where no trusted organization exists building a blockchain
requires a protocol that creates consensus across the community about the current, correct state of the blockchain.
There are many different protocols for building consensus based on requirements related to performance, scalability,
consistency, threat model, and failure model. Traditionally, protocols that achieve consensus for arbitrary
(i.e. Byzantine) faults require several rounds of explicit voting. While this approach provides high throughput and
finality of commits, its reliance on a well-known, generally static voting block and the heavy use of communication
(multiple rounds of N^2 messages) limit its use to reletively small, localized, and stable applications. 

Bitcoin introduced a new method for achieving consensus, called Nakamoto consensus, that supports broad participation
from a dynamic group of participants. Nakamoto consensus relies on uncoordinated, distributed leader election that is
efficient for communication, but often very computationally intense. For example, Bitcoin elects the leader using
``proof-of-work'', a race to solve a cryptographic puzzle through repetitive ``guess and test''. The first participant
to solve the puzzle is granted unilateral authority to propose the next block that will be added to the blockchain.
While Nakamoto consensus provides a global agreement protocol that is highly resilient to faults, the computational
characteristics of proof-of-work drive processing to custom hardware that can consume gigawatts of power~\cite{}.

This paper describes a new leader election algorithm designed for Nakamoto consensus, called ``Proof of Elapsed Time''
or PoET, that elects leaders through a simple computation in a trusted execution environment provided by commodity
processors.  PoET is based on the observation that proof-of-work, at its core, is a method for generating and enforcing
random wait times--proof-of-work elects as leader the first participant to solve the cryptographic puzzle, a time that
is a random interval drawn from an exponential distribution. With PoET, participants use Intel's Software Guard
Extensions (SGX)~\cite{} to generate and then enforce a random wait time without the cryptographic puzzle. Where
proof-of-work can verify correctness by providing the solution to the puzzle, PoET uses an attestation of correctness
that is provided by the trusted execution environment; any participant can verify that the wait time was generated
correctly and enforced.  In this way PoET preserves the fairness and integrity of leader election provided by
proof-of-work without high cost of electricity and custom hardware. This allows deployment of highly-resilient, fault
tolerant applications for a dynamic community using commonly available computing resources.

% -----------------------------------------------------------------
%% Unfortunately, there are a variety of limitations that handicap the block chain in its current
%% instantiation and prevent it from being used for other (potentially) useful purposes. We focus on
%% leveraging Intel's SGX environment [AGJS13] to re-engineer the proof of work to a proof of wait,
%% creating a new lottery-based consensus protocol. This will provide greater efficiency and reduce
%% the amount of energy used and high costs that we currently see.

%% The key concept of the proof of wait is a guaranteed wait time, provided through the Trusted
%% Execution Environment (TEE) that we get with SGX. Each \miner" will request a wait time from
%% the TEE, and the miner with the lowest wait time \wins," thus getting to create the next block.
%% In order to prove that the node did indeed wait the claimed time, an attestation is also created
%% within the TEE. This validation will be done using the Enhanced Privacy ID (EPID) [BL07, BL09]
%% scheme

For the purpose of achieving distributed consensus efficiently, a good leader election function has several
characteristics:

\begin{itemize}
\item Fairness:
  The function should distribute leader election across the broadest possible population of participants.

\item Investment:
  The cost of controlling the leader election process should be proportional to the value gained from it. 

\item Verification:
  It should be relatively simple for all participants to verify that the leader was legitimately selected. 
\end{itemize}

The PoET leader election algorithm meets the criteria for a good lottery algorithm. It randomly distributes leadership
election across the entire population of validators with distribution that is similar to what is provided by other
lottery algorithms. The probability of election is proportional to the resources contributed (in this case, resources
are general purpose processors with a trusted execution environment). An attestation of execution provides information
for verifying that the certificate was created within the TEE (and that the validator waited the allotted
time). Further, the low cost of participation increases the likelihood that the population of validators will be large,
increasing the robustness of the consensus algorithm.
