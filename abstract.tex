%% -*- fill-column: 120; -*-
\section{Abstract}
\label{sec_abstract}

Consensus is the process of building agreement amongst a group of mutually distrusting participants. There are many
different protocols for building consensus based on requirements related to performance, scalability, consistency,
threat model, and failure model. Traditionally, protocols that achieve consensus for arbitrary (i.e. Byzantine) faults
require several rounds of explicit voting. While this approach provides high throughput and finality of commits, its
reliance on a well-known, generally static voting bloc and the heavy use of communication (multiple rounds of N^2
messages) limit its use to reletively small, localized, and stable applications. Bitcoin introduced a new method for
achieving consensus, called Nakamoto consensus, that supports broad participation from a very dynamic group of
participants. Nakamoto consensus relies on uncoordinated, distributed leader election that is often very computationally
intense. For example, Bitcoin uses ``proof of work'', a computation to solve a cryptographic puzzle through repetitive
``guess and test''; a process that consumes gigawatts of power. 

This paper describes a new leader election algorithm designed for Nakamoto style consensus, called ``Proof of Elapsed
Time'' or PoET, that elects leaders through a simple computation in a trusted execution environment provided by
commodity processors. PoET preserves the fairness and integrity of leader election provided by proof-of-work without
high cost of electricity and custom hardware. This allows deployment of highly-resilient, fault tolerant applications
for a dynamic community using commonly available computing resources. 

